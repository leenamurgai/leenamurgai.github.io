%%%%%%%%%%%%%%%%%%%%%%%%%%%%%%%%%%%%%%%%%%%%%%%%%%%%%%%%%%%%%%%%%%%%%%%%%%%%
%%%%%%%%%%%%%%%%%%%%%%%%%%%%%%%%%%%%%%%%%%%%%%%%%%%%%%%%%%%%%%%%%%%%%%%%%%%%
%%%%%%%%%%%%%%%%%%%%%%%%%%%%%%%%%%%%%%%%%%%%%%%%%%%%%%%%%%%%%%%%%%%%%%%%%%%%
\section{Individual fairness}\label{app_AIF360_IF}

%%%%%%%%%%%%%%%%%%%%%%%%%%%%%%%%%%%%%%%%%%%%%%%%%%%%%%%%%%%%%%%%%%%%%%%%%%%%
%%%%%%%%%%%%%%%%%%%%%%%%%%%%%%%%%%%%%%%%%%%%%%%%%%%%%%%%%%%%%%%%%%%%%%%%%%%%
\subsection{Consistency}

\begin{lookbox}
\lbtitle{Exercise: Consistency score}
Use AIF360 to calculate consistency for the Statlog (German Credit) data and your model from chapter 3 which classified loan applicants as presenting good or bad credit risks. See section 7 of the jupyter notebook \href{https://github.com/leenamurgai/mitigatingbiasml/blob/master/code/source/mbml-german.ipynb}{\texttt{mbml\_german.ipynb}}
\end{lookbox}

The consistency metric in AIF360 uses Euclidean distance by default, but does allow the user to specify their own distance metric. 
